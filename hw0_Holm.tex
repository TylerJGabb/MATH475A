\documentclass[12pt]{article}
\usepackage{lmodern}  % for bold teletype font
\usepackage{amsmath}  % for \hookrightarrow
\usepackage{xcolor} 
\usepackage{amssymb}
\usepackage{fdsymbol}
\usepackage{listings}

\lstset{language=Python,basicstyle=\ttfamily, columns=fullflexible,frame=single,breaklines=true,postbreak=\mbox{\textcolor{red}{$\hookrightarrow$}\space}}


\begin{document}

\noindent
Adam Holm\\
Math 475A\\
HW 1

\section*{Problem 1}

Use $\int_a^b f(x)g(x)dx = f(c) \int_a^b g(x)dx$, when $f \in \mathbf{C}[a,b]$ and $g(x)$ does not change sign in $[a,b]$, and $c \in [a,b]$ to show that when $F \in \mathbf{C}^1[a,b]$, \\ $\frac{F(b)-F(a)}{b-a} = F'(c)$ for some $c \in (a,b)$. \\

\noindent
Let $g(x) = w$ where $w$ is some constant.\\
Then $\int_a^b f(x)g(x)dx = w \int_a^b f(x)dx = w f(c)(b-a)$\ldots (not done) 

\section*{Problem 2}
\noindent
Program written in Python $2.7.12$

\begin{lstlisting}

import sys
import re
	

def gcd(a, b):
	# doesn't sort a and b values so that a > b,
	# as results are the same regardless, at this stage
	aModB = a % b # aModB = a modulo b
	if aModB == 0:
		# case where solution is found
		return b
	else:
		# if solution isn't found yet
		if aModB >= b:
			# x and y are values for recursive
			# call of the function.
			# Sorting values for proper calculation
			# in this if statement.
			x = aModB
			y = b
		else:
			x = b
			y = aModB
		return gcd(x,y)
		
	

def main():
	print 'Program will calculate gcd(a,b)'
	cont = 'Yes'
	# while loop to make it possible to calculate multiple gcds
	while re.match('yes', cont, flags=re.IGNORECASE):
		# take two integers
		# checking for proper input in while loop and all nested loops
		while True:
			try:
				a = int(raw_input("Value for a: "))
				if a < 0:
					print 'a needs to be positive'
					continue
				break	
			except:
				print 'a needs to be a positive integer'
			
		while True:
			try:
				b = int(raw_input("Value for b: "))
				if b < 0:
					print 'b needs to be positive'
					continue
				break
			except:
				print 'b needs to be a positive integer'

		# if input is 0, print 0 instead of calling gcd function to avoid errors of calculating with 0
		if ( a == 0 ) | (b == 0):
			print 'gcd({!r},{!r}) = 0'.format(a,b)
		else:
			print 'gcd({!r},{!r}) = {!r}'.format(a,b,gcd(a,b))
 		cont = str(raw_input("Would you like to calculate another gcd(a,b)? ('yes' to continue) "))

main()

\end{lstlisting}

\newpage

Output:

\begin{lstlisting}
Program will calculate gcd(a,b)
Value for a: 455
Value for b: 66
gcd(455,66) = 1

Value for a: 1001
Value for b: 21
gcd(1001,21) = 7

Value for a: 47
Value for b: 3
gcd(47,3) = 1
 



\end{lstlisting}




\end{document}